%!TEX root = ../talk.tex
\chapter{Overview and Comparison of P2P File Synchronization and Storage Solutions}
\markboth{Overview and Comparison of P2P File Synchronization and Storage Solutions}{}
\chaptauthors{Josua Fr\"ohlich, Simon Ruesch}

\Kurzfassung{
This is where the abstract will go.
}

\newpage

% the table of contents abc
\minitoc

\newpage

\section{Introduction and Problem Statement} Today, file synchronization and
storage is demanded more than ever. The average user owns
multiple devices that are continuously connected to the internet and uses the
device which is the most practical for his current situation. As a consequence,
users require their files to be up-to-date at all times regardless of which
device the user wants to access the files on. A second aspect is that there is
an ever increasing demand for systems which allow file sharing between users,
such that multiple people can work on the same files independently, whether at the
same time or not.

In general, the existing file synchronization and storage systems can be
categorized into two groups: Client-Server based distributed file storage
systems (also sometimes referred to as Cloud storage systems) and Peer-to-Peer
file storage systems. Since giving a comprehensive treatment to both of those
categories would be beyond the scope of this seminar thesis, the focus here lies on
Peer-to-Peer file storage systems, henceforth called P2P file storage systems.

This seminar thesis aims to give an overview over the current available P2P file
storage systems, to categorize them according to their different aspects and to
compare their assets and drawbacks. The findings will also be compared to a
popular distributed file storage system.

\section{Background} This section gives a short basic overview over the most
important topics concerning P2P systems, file storage, and the special
requirements of P2P file storage systems compared to those of traditional and
distributed file storage.

\section{Approach} This section describes the criteria we chose to classify the
different file storage systems.

\subsection{Criteria}
\begin{itemize}
\item Type of Distribution: Is the system only a Library or a dedicated software client
\item Payment Scheme: Is the system free or paid? Who gets paid what under which circumstances?
\item Source code availability: Is the system open-source or closed source?
\item Peering Scheme: How are peers connected, how do they communicate and what messages do they interchange (pure, hybrid, centralized, structured peering)
\item Encryption: Are the communication between peers and the files stored encrypted and with what algorithm?
\item Trust: How is trust established between peers?
\item Integrity: How are files and communication protected against fraudulent users and attacks?
\item Userbase: How many users does the system currently have? Is the system intended for businesses or private users?
\item Fairness: How is the workload distributed in the system especially considering the heterogeneous hardware used by the peers? 
\item Fault-tolerance and Availability: How does the system guarantee that files are available to the users at all time? 
\item Status: In what state is the system? Is it available for use, a scientific prototyp or cancelled?
\end{itemize}

\section{Comparisons} In this section, we classify and compare four exemplary
P2P file storage systems according to the classification criteria given in the
preceding section. We also compare the P2P file storage to a popular distributed
file storage system to find similarities and differences. At the end, we give a
summary of the findings and highlight those findings in a table comparing the
file storage systems with each other.

\subsection{Storj}

\subsection{Hive2Hive / PeerWasp}

\subsection{BitTorrent Sync}

\subsection{AeroFS}

\subsection{Dropbox}

\section{Summary and Conclusions} In this section, we provide a summary about
the topics mentioned above and draw conclusions about the current state of the
available P2P file storage systems and what the future could hold for P2P file
storage.

\begin{thebibliography}{99}

\bibitem {aerofs} \emph{AeroFS} \url{https://www.aerofs.com/}, March, 2015.

\bibitem {bitcoin} S. Nakamoto: \emph{Bitcoin: A Peer-to-Peer Electronic Cash System;} \url{https://bitcoin.org/bitcoin.pdf}, October, 2008.

\bibitem {bittorrentsync-2} J. Farina, M. Scanlon, M. Kechadi: \emph{BitTorrent Sync: First Impressions and Digital Forensic
Implications;} Proceedings of the First Annual DFRWS Europe, \url{http://ac.els-cdn.com/S1742287614000152/1-s2.0-S1742287614000152-main.pdf?_tid=10ddb6e2-ce69-11e4-9019-00000aacb35d&acdnat=1426791318_6677afbe19d521d323605261c1d19809}, Volume 11, Supplement 1, Pages S77\textendash S86, Volume 11, May, 2014.

\bibitem {box2box} A. Lareida, T. Bocek, S. Golaszewski, C. L\"uthold, M. Weber: \emph{Box2Box - A P2P-based File-Sharing and Synchronization Application;} \url{http://www.csg.uzh.ch/csg/live/teaching/FS13/p2p/challenge/P2P2013DP_017.pdf}, September, 2013.

\bibitem {hive2hive} \emph{Hive2Hive: Open-Source Library for P2P-based File Synchronization and Sharing;} \url{http://hive2hive.com/}, March, 2015.

\bibitem {metadisk} S. Wilkinson, J. Lowry: \emph{Metadisk: Blockchain-based decentralized file storage application;} \url{http://metadisk.org/metadisk.pdf}, December, 2014.

\bibitem {p2pfswu} C. Wu: \emph{Peer-to-Peer Networks;} \url{http://www.csie.nuk.edu.tw/~wuch/course/csf641/csf641-04-storage.pdf}, March, 2015.

\bibitem {p2pfskangasharju} J. Kangasharju: \emph{Peer-to-Peer Networks Chapter 4: Peer-to-Peer Storage;} \url{http://www.cs.helsinki.fi/u/jakangas/Teaching/P2P/P2P-04-Storage.pdf}, March, 2015.

\bibitem {storjPDF} S. Wilkinson: \emph{Storj A Peer-to-Peer Cloud Storage Network;} \url{http://storj.io/storj.pdf/}, December, 2014.

\bibitem {bittorrentsync} \emph{Sync;} \url{https://www.getsync.com/}, March, 2015.

\end{thebibliography}